\documentclass[12pt, a4paper]{article}
\usepackage[margin=1in]{geometry}
\usepackage[utf8x]{inputenc}
\usepackage{indentfirst} %indentace prvního odstavce
\usepackage{mathtools}
\usepackage{amsfonts}
\usepackage{amsmath}
\usepackage{amssymb}
\usepackage{graphicx}
\usepackage{enumitem}
\usepackage{subfig}
\usepackage{float}
\usepackage[czech]{babel}
\usepackage{mathdots}
\usepackage{slashbox}

\begin{document}
\begin{center}
\large NMAG436 - HW4

\normalsize Jan Oupický
\end{center}
\vspace{1\baselineskip}

\section{}
Let $L \coloneqq \mathbb{F}_p(V_{w_a})$. $w_a$ is a short WEP by definition for every $p$ and $a$. 
\\

First let $p=2$: 

$w_0 = y^2+x^3, w_1 = y^2 + x^3 + 1$ with partial derivatives $\frac{\partial w_0}{\partial x} (x,y) = x^2, \frac{\partial w_0}{\partial y} (x,y) = 0, \frac{\partial w_1}{\partial x} (x,y) = x^2, \frac{\partial w_1}{\partial y} (x,y) = 0$. We see that $V_{w_0}(\mathbb{F}_2) = \{(0,0), (1,1)\}, V_{w_1}(\mathbb{F}_2) = \{(1,0),(0,1)\}$. We see that $w_0$ is not smooth at $(0,0)$ and $w_1$ at $(0,1)$ therefore they are not smooth. 

Theorem 8.4 tells us that $L$ is an EFF iff $w$ is smooth therefore $L$ is not an EFF. Proposition 8.3 5) tells us that the only other option is that the genus of $L$ is 0.
\\

Now let $p>2$:

Let $f \coloneqq x^3+a \in \mathbb{F}_p[x]$. We want to know for which $a$ is $f$ separable. By definition we want to know when $\text{GCD}_{\mathbb{F}_p[x]}(f,f')=1$. Since $f' = 3x^2$ for every $a$, we can see that $f,f'$ are not coprime iff $a = 0$ (if $p=3$ then $f'=0$ and still: $f$ is separable iff $a\neq0$). From that we see:


If $a=0$ then $w_a$ is not smooth by 3.12. 3) which implies that the genus of $L$ is 0 (same reasoning as in the case $p=2$).

If $a\neq 0$ then $w_a$ is smooth and by 8.4 the genus of $L$ is 1. 

\section{}
Let $L=\mathbb{F}_5(V_w)$ (i.e. $L$ is given by $w(\alpha,\beta)=0$ where $\alpha = x + (w), \beta = y + (w)$).
\\

We calculate the partial derivatives of $w$: $\frac{\partial w}{\partial x}(x,y)=y-3x^2+1, \frac{\partial w}{\partial y}(x,y) = 2y+x$. By substituting the point $(1,2)$ we get that both derivatives are equal to 5 which is 0 since we are in $\mathbb{F}_5$. Therefore $w$ is singular at $(1,2)$.

Now we need a shifted polynomial which gives us the same $L$. As in previous exercices we use translation $\tau_\gamma$ given by a vector $\gamma \coloneqq (1,2)$. We denote the new polynomial which is singular at $(0,0)$ by $w'$. As before $w' \coloneqq \tau_{\gamma}^*(w) = w(x+1,y+2) = y^2+xy+4x^3+2x^2$ and also we get elements $(u,t) = \tau_{-\gamma}(\alpha,\beta) = (\alpha-1,\beta-2)$ (following the proof of 5.8, 5.5 and 3.10). Now we now that $L$ is also given by $w'(u,t)=0$.

We have now satisfied the conditions assumed in the one implication in proof of 8.4 and we can follow it. So we define $s \coloneqq \frac{t}{u}$. As in the proof we know $w'(u,t)=0 \implies \frac{w'(u,t)}{u^2} = 0 \implies 0 = s^2+s-u+2 \iff u = s^2+s+2 \in \mathbb{F}_5(s)$ and from the definition of $s$ we get $t = su = s(s^2+s+2)\in \mathbb{F}_5(s)$. Which means that $L = \mathbb{F}_5(\alpha,\beta)=\mathbb{F}_5(u,t) = \mathbb{F}_5(s)$. 

But from the definitions: $s = \frac{t}{u} = \frac{\alpha-1}{\beta-2} = \frac{x-1 + (w)}{y-2 + (w)} \in L$ we see that the element we are looking for is $\in \mathbb{F}_5(x,y)$ therefore we let $s$ be actually $\frac{x-1}{y-2}$.

\section{}
Let $L = \mathbb{F}_5(V_f)$ (i.e. $L$ is given by $f(\alpha,\beta)=0, \alpha = x+(f), \beta= y +(f) $) where $f = y^2-(x^3-2) \in \mathbb{F}_5[x,y]$ and denote $\bar{f} \coloneqq (x^3-2)$. We see that $f$ is a (short) WEP therefore absolutely irreducible by 4.9. By calculating $\bar{f}' = 3x^2$ and $\text{GCD}_{\mathbb{F}_5[x]}(\bar{f},\bar{f}') = 1$ we see that $\bar{f}$ is separable in $\mathbb{F}_5[x]$ therefore $f$ is smooth (at $V_f$). We have satisfied the assumptions of theorem 8.4 and so we have proved that $L$ is EFF. 

\begin{enumerate}[label=(\alph*)]
\item Using definition of $E \coloneqq E(\mathbb{F}_5) = V_f(\mathbb{F}_5) \cup \{\infty\}$ we have to find all roots of $f$ in $\mathbb{F}_5$ so thats (25 combinations). 

We calculate that $V_f(\mathbb{F}_5) = \{(1,2),(1,3),(2,1),(2,4),(3,0)\} \implies \\E = \{(1,2),(1,3),(2,1),(2,4),(3,0),\infty\}$. We know that $E$ is finite and an abelian group therefore is it cyclic (and therefore isomorphic to $\mathbb{Z}_6)$. By Lagrange theorem  we know that $E$ can have elements only of orders 1 (neutral element which is $\infty$ by definition of $E$), 2, 3 and 6 (a generator). We want to look for $\gamma \in V_f(\mathbb{F}_5)$ such that $\gamma \oplus \gamma \neq \infty$ (not of order 2) and $\gamma \oplus \gamma \neq \ominus \gamma \iff \gamma \oplus \gamma \oplus \gamma = \infty$ (not of order 3).

If we straight up use the formulas given by theorem 8.8 (where $a_1=a_2=a_3=a_4 = 0, a_6 = -2=3$)  we see that $\gamma = (1,2)$ is of order 6 since:
\begin{gather*}
\text{using 8.8 1): }\ominus \gamma = (1, -2) = (1,3) \implies  \gamma \neq \ominus \gamma\\
\delta \coloneqq \gamma \implies \gamma \neq \ominus \delta \text{ assumption of 2)}\\
\text{using 8.8 2): } \mu = \gamma \oplus \gamma\\
\lambda \coloneqq \frac{3 \cdot (1)^2}{2 \cdot 2} = \frac{3}{4} = \frac{3}{-1} = -3 = 2\\
\implies \mu_1 = -1-1+2^2 = -2+4 = 2 \\
\implies \mu_2 = 2(1-2)-2 = -2 - 2 = - 4 = 1 \implies\\
\gamma \oplus \gamma = (2,1) \neq \infty,\ominus \gamma
\end{gather*}

Therefore $(1,2)$ is a generator.

(I haven't discovered the mentioned geometrical ideas in the proof that would help me solve this more easily.)

\item Let $D \coloneqq \sum_{\gamma \in E(\mathbb{F}_5)} 1P_\gamma$. By definition $deg(D) = \sum_{\gamma \in E(\mathbb{F}_5)}1 deg_{\mathbb{F}_5}(P_\gamma)$. Since we have shown that $w$ is smooth then by 8.3 4) we know that $deg_{\mathbb{F}_5}(P_\gamma) = 1, \forall \gamma \in E(\mathbb{F}_5)$. We have shown that $L$ is EFF that means $L$ is full constant and genus is 1. So we can use Corollary 7.6 2) since $6 \geq 2 - 1 = 1 \implies l(D) = deg(D) + 1 - 1 = 6$. By definition $l(D) = dim_{\mathbb{F}_5}(\mathcal{L}(D))$ so the $\mathbb{F}_5$-dimension of $R$ is 6.

Using lemma 6.2 where $A \coloneqq \underline{0}, B \coloneqq D$ ($D \geq \underline{0}$ by definition) we get that $\mathcal{L}(\underline{0}) \subseteq \mathcal{L}(D)$. Using observation B 5) we get that $\mathcal{L}(\underline{0}) = \tilde{\mathbb{F}}_5 \stackrel{L\text{ is full constant}}{=} \mathbb{F}_5$. Therefore we can see that for example 2 is a nonzero element of $\mathbb{F}_5 \subseteq R$.

\end{enumerate}
\end{document}