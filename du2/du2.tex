\documentclass[12pt, a4paper]{article}
\usepackage[margin=1in]{geometry}
\usepackage[utf8x]{inputenc}
\usepackage{indentfirst} %indentace prvního odstavce
\usepackage{mathtools}
\usepackage{amsfonts}
\usepackage{amsmath}
\usepackage{amssymb}
\usepackage{graphicx}
\usepackage{enumitem}
\usepackage{subfig}
\usepackage{float}
\usepackage[czech]{babel}
\usepackage{mathdots}
\usepackage{slashbox}

\begin{document}
\begin{center}
\large NMAG436 - HW1

\normalsize Jan Oupický
\end{center}
\vspace{1\baselineskip}

\section{}
By definition of $L$ being AFF given by $f(\alpha,\beta) = 0$ we know that $f$ is irreducible in $\mathbb{Q}$ and $L=\mathbb{Q}(\alpha,\beta)$.
\begin{enumerate}[label=(\alph*)]
\item Using proposition 4.7 we know that $\alpha$ is transcendental over $\mathbb{Q}$ because $deg_y(f)=2>0$ and also that $[\mathbb{Q}(\alpha,\beta):\mathbb{Q}(\alpha)]=2$. In other words, the basis A has 2 elements. Let $A=(1, \beta)$ (first element of basis A is $1$ and the second is $\beta$. We can see that $1$ and $\beta$ are linearly independent: 
\begin{gather*}
c_1,c_2 \in \mathbb{Q}(\alpha) :1c_1+\beta c_2 = 0 \stackrel{?}{\iff} c_1 = 0 = c_2\\
1c_1+\beta c_2 = 0 \iff \beta c_2 = -c_1 \stackrel{c_2 \neq 0}{\iff} \beta = \frac{-c_1}{c2} \in \mathbb{Q}(\alpha)\\
\implies \text{contradiction with }[\mathbb{Q}(\alpha,\beta):\mathbb{Q}(\alpha)]=2 \implies c_1 = 0 = c_2
\end{gather*}
We know the basis has 2 elements and therefore $A$ is a basis.
\item Using the same reasoning ($\beta$ is also transcendental over $\mathbb{Q}$) using symetry (we can just replace $y$ with $x$) we get that $[\mathbb{Q}(\alpha,\beta):\mathbb{Q}(\beta)]=deg_x(f)=3$. Therefore the basis $B$ has 3 elements. Let $B = (1, \alpha, \alpha^2)$. Following the proof of lemma 4.6 let $m(x) \coloneqq f(x, \beta) \in \mathbb{Q}(\beta)[x]$. The proof shows that $m(x) = -x^3-2x^2-1+\beta^2$ is a minimal polynomial of $\alpha$ over $\mathbb{Q}(\beta)$. 

We want to prove that elements of $B$ are linearly independent $\iff$ ($b_0+b_1\alpha + b_2\alpha^2 = 0$ where $b_1,b_2,b_3 \in \mathbb{Q}(\beta) \iff b_1 = 0, b_2 = 0, b_3 = 0)$. Assume that there exists $b_1 + b_2\alpha + b_3\alpha^2 = 0$ where at least one $b_i \neq 0$. That would mean that $\alpha$ is a root of a non zero polynomial $\in \mathbb{Q}(\beta)[x]$. Since $m(x)$ is a minimal polynomial of $\alpha \implies m(x) | b_1+b_2x+b_3x^2$  which is impossible since $deg_x(m) = 3 > deg_x(b_1+b_2x+b_3x^2)$. Thats a contradiction therefore $1,\alpha, \alpha^2$ are linearly independent and form a basis.

\item We will use equalities given by $f(\alpha, \beta) = 0 \implies \beta^2 = \alpha^3+2\alpha^2+1, \alpha^3 = \beta^2-2\alpha^2-1$.
\begin{gather*}
[\alpha^3\beta^3]_A: \alpha^3\beta^3 = \alpha^3((\alpha^3+2\alpha^2+1)\beta) = \beta(\alpha^6+2\alpha^5+\alpha^3), (\alpha^6+2\alpha^5+\alpha^3) \in \mathbb{Q}(\alpha)\\
A=(1,\beta) \implies [\alpha^3\beta^3]_A = (0, \alpha^6+\alpha^5+\alpha^3)\\
[\alpha^3\beta^3]_A: \alpha^3\beta^3 = (\beta^2-2\alpha^2-1)\beta^3 = 1(\beta^5-\beta^3) + \alpha^2(-2\beta^3)\\
B = (1, \alpha, \alpha^2) = (\beta^5-\beta^3, 0, -2\beta^3)
\end{gather*}
\end{enumerate}

\section{}
\begin{enumerate}[label=(\alph*)]
\item Let $\gamma \coloneqq (1,2)^T \in \mathbb{Q}^2$. By calculating derivatives $\frac{\partial f}{\partial x} (\gamma) = -7, \frac{\partial f}{\partial y} (\gamma) = 4$ we get $t_{\gamma}(f) = -7(x-1)+4(y-2) = -7x+4y-1$. Let $(a_1,a_2) \coloneqq (-7,4)$ from lemma 5.7. Using lemma 5.7 we know we are looking for $\sigma \coloneqq \theta_A \tau_{-\gamma}$ where $A$ is a regular rational matrix of the form $A = \begin{pmatrix}b_1 & b_2 \\ -7 & 4\end{pmatrix}$. From definition of $\sigma$ we get $\sigma(1,2)^T = A(0,0)^T = 0$.\\
Let $(b_1,b_2)=(1,0)$ for example. We can see that $A$ is regular. We now have to calculate the polynomial $\hat{f}$ following the proof of lemma 5.7. We will first calculate $\tilde{f}(x,y)$.
\begin{gather*}
p(x,y) \coloneqq \tilde{f}(x-1,y-2) = f(x,y)-t_\gamma(f)= y^2-x^3-2x^2+7x-4y\\
\tilde{f}(x,y) \stackrel{\text{substitution}}{=} p(x+1,y+2) = y^2-x^3-5x^2\\
A^{-1} = \begin{pmatrix}
1 & 0\\
\frac{7}{4} & \frac{1}{4}
\end{pmatrix} \implies \theta_{A^{-1}}^*(p(x,y)) = p(x, \frac{7}{4}x+\frac{1}{4}y)\\
\hat{f}(x,y) = \theta_{A^{-1}}^*(\tilde{f}(x,y))=\tilde{f}(x, \frac{7}{4}x+\frac{1}{4}y) = -x^3-\frac{31 x^2}{16}+\frac{7 x y}{8}+\frac{y^2}{16}\\
\implies h(x)\coloneqq -x^3-\frac{31 x^2}{16}, g(x,y) \coloneqq \frac{7 x}{8}+\frac{y}{16}\\
\sigma^*(p(x,y)) = p(x-1, -7x+4y-1)
\end{gather*}
We can check $\sigma^*(\hat{f}+y) = \sigma^*(h(x)+yg(x,y)+y) =  f$ . The desired map is $\sigma$, where $A$ is for example written above.
\item From the lemma 5.7 we know $\sigma$ has the form $\sigma \coloneqq \theta_A \tau_{-\gamma}$, where $A = \begin{pmatrix}b_1 & b_2 \\ -7 & 4\end{pmatrix}$ is regular $\iff det(A) \neq 0$. We will now calculate $\sigma(0,0)^T:$
\begin{gather*}
A = \begin{pmatrix}b_1 & b_2 \\ -7 & 4\end{pmatrix}, \sigma(0,0)^T = A\begin{pmatrix}
-1 \\ 2
\end{pmatrix} = -1 \begin{pmatrix}
b_1 \\ -7
\end{pmatrix} -2 \begin{pmatrix}
b_2 \\ 4
\end{pmatrix} = \begin{pmatrix}
-b_1 - 2b_2\\ -1
\end{pmatrix}\\
\text{we want }det(A) \neq 0 \iff 4b_1 + 7b_2 \neq 0 \iff b_1 \neq \frac{-7}{4}b_2 \\
\text{in other words } \begin{pmatrix}
b_1 \\
b_2
\end{pmatrix} \in \mathbb{Q}^2 \setminus Span_{\mathbb{Q}}\begin{pmatrix}
-7 \\ 4
\end{pmatrix}
\end{gather*}
To sum it up. There exists $\sigma$ satisfying given conditions if and only if $\mathbf{a} = \begin{pmatrix}
-b_1-2b_2 \\ -1
\end{pmatrix}$ where $(b_1,b_2)^T$ is not in the span of the vector $(-7, 4)$.
\end{enumerate}


\section{}
$\nu$ is a normalized DV of $L$, by definition that means there exists $R \subseteq L$ DVR with a maximal ideal $M$ and uniformizing element $t \in M: (t)=M$ and $\nu \coloneqq \nu_t$. We also know that $R = \theta_M$ (notation). We also know that $\nu(\alpha-1)>0,\nu(\beta-2)>0$.

Theorem 5.8 tells us (using $\gamma = (1,2)$) that there exists exactly one $P \in \mathbb{P}_{L/K}$ s.t. $\nu_P(\alpha-1)>0,\nu_P(\beta-2)>0$. By definition $P = (p), p \in P$ and $\nu_P \coloneqq \nu_p$. Theorem 2.15 (2) tells us that $\theta_M = \theta_P \implies \nu = \nu_P$. So from now on we can work with this uniquely defined DV $\nu$.

Using 2) from 5.8 we can assume that $l(\gamma)=0$ since we are only looking for $l_0,l_1,l_2$ when $\nu(l(\alpha,\beta))>0$. Let $(u,v) \coloneqq \bar{\sigma}(\alpha,\beta)$ as in the proof of 5.8 with $(b_1,b_2) = (1,0)$. 

In the proof we see that $l(\alpha,\beta) \in Span(u,v)$ if $l(\gamma)=0$, that means $\exists k_1,k_2 \in \mathbb{Q}: l(\alpha, \beta) = k_1u + k_2v$. Using definition of $\nu_p$ we see that $(p | l(\alpha,\beta) \iff p| k_1u+k_2v) \implies \nu(l(\alpha, \beta))=\nu(k_1u+k_2v)$. 

We also know using proposition 5.5 that $\nu(u)=1$ and $\nu(v)=m>1$ where $m \coloneqq mult(h) = mult(-x^3-\frac{31 x^2}{16})$ from exercise 2 $\implies \nu(v)=2$.

Now using properties of valuation (assuming $k_1,k_2 \neq 0$): $\nu(k_1u)=\nu(k_1)+\nu(u) = 0 + 1 = 1, \nu(k_2v)=\nu(k_2)+\nu(v) = 0 + 2 = 2 \stackrel{2.13}{\implies} \nu(k_1u+k_2v) = min(\nu(k_1u),\nu(k_2v)) = min(1,2) = 1$.

$\nu(k_1u+k_2v) = 1 \iff k_1 \neq 0, k_2 \in \mathbb{Q}$. If $k_2=0$ then we have $\nu(k_1u)=1$ as stated above.

Similarly we can see that $\nu(k_1u+k_2v) = 2 \iff k_1 = 0, k_2 \neq 0$. Also by choosing any combination of $k_1,k_2 \in \mathbb{Q}$ we cannot get $\nu(k_1u+k_2v) = 3$.

Now we have constraints on $k_1,k_2 \in \mathbb{Q}$ and we can "transform" these elements ($k_1u+k_2v$) to a $\alpha, \beta$ representation using $\bar{\sigma}$. $\bar{\sigma}^{-1} \coloneqq \tau^{-1}_{-\gamma} \theta^{-1}_{A} = \tau_{\gamma}\theta_{A^{-1}}$ with the values from exercise 2. We see that $\bar{\sigma}^{-1}(k_1u+k_2v) = k_1+k_1\alpha+2k_2+\frac{7}{4}k_2\alpha + \frac{1}{4}k_2\beta $. When we rearange these elements we can form $l(\alpha,\beta) = l_0 + l_1\alpha + l_2\beta$ where (using vector notation for simplicty):
\begin{gather*}
\begin{pmatrix}
l_0\\
l_1\\
l_2\\
\end{pmatrix} = k_1\begin{pmatrix}
1\\
1\\
0\\
\end{pmatrix} + k_2\begin{pmatrix}
2\\
\frac{7}{4}\\
\frac{1}{4}\\
\end{pmatrix}
\end{gather*}

We defined conditions on $k_1,k_2$ based on valuation value and now we have expressed $l_0,l_1,l_2$ using $k_1,k_2$. We also need $l(\gamma)=l(1,2)=0$. To sum it up:
\begin{gather*}
\text{in both cases we have condition: } l(\gamma)=l_0+1l_1+2l_2 = 0\\
\nu(l(\alpha,\beta)) = 1 \iff \begin{pmatrix}
l_0\\
l_1\\
l_2\\
\end{pmatrix} = k_1\begin{pmatrix}
1\\
1\\
0\\
\end{pmatrix} + k_2\begin{pmatrix}
2\\
\frac{7}{4}\\
\frac{1}{4}\\
\end{pmatrix}, k_1 \in \mathbb{Q}\setminus\{0\}, k_2 \in \mathbb{Q}\\
\nu(l(\alpha,\beta)) = 2 \iff \begin{pmatrix}
l_0\\
l_1\\
l_2\\
\end{pmatrix} = k_2\begin{pmatrix}
2\\
\frac{7}{4}\\
\frac{1}{4}\\
\end{pmatrix}, k_2 \in \mathbb{Q}\setminus\{0\}
\end{gather*}
and $\nu(l(\alpha,\beta)) \neq 3$ for any values $l_0,l_1,l_2$. We also see that $\nu(l(\alpha,\beta)) \neq 2$ for any values $l_0,l_1,l_2$ because $k_2(2+\frac{7}{4}+\frac{1}{4})\neq 0$ for any $k_2$ nonzero.
\end{document}

