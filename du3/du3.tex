\documentclass[12pt, a4paper]{article}
\usepackage[margin=1in]{geometry}
\usepackage[utf8x]{inputenc}
\usepackage{indentfirst} %indentace prvního odstavce
\usepackage{mathtools}
\usepackage{amsfonts}
\usepackage{amsmath}
\usepackage{amssymb}
\usepackage{graphicx}
\usepackage{enumitem}
\usepackage{subfig}
\usepackage{float}
\usepackage[czech]{babel}
\usepackage{mathdots}
\usepackage{slashbox}

\begin{document}
\begin{center}
\large NMAG436 - HW3

\normalsize Jan Oupický
\end{center}
\vspace{1\baselineskip}

\section{}
By calculating $\frac{\partial f_1}{\partial x} (x,y) = 3x^2+1=x^2+1, \frac{\partial f_1}{\partial y} (x,y) = 2y=0$ we see that $f_1$ is singular at $(1,1)$ because $\frac{\partial f_1}{\partial x} (1,1)= 1+1=0, \frac{\partial f_1}{\partial y} (x,y)=0$. To be able to use lemma 5.11 we need to express $L_1=\mathbb{F}_2(\alpha,\beta)$ using a polynomial $f'$ that is singular at $(0,0)$. We define $f'=\tau_{(1,1)}^{*}(f) = y^2+x^3+x^2$. Now can say that $L_1$ is AFF over $\mathbb{F}_2$ given by $f'(u,t)=0$ where $(u,t)=\tau_{(1,1)}(\alpha,\beta) = (\alpha+1,\beta+1)$. Now we apply lemma 5.11 that tells us that $\frac{u}{t}=\frac{\alpha+1}{\beta+1} \notin \prescript{}{f'}{\mathcal{O}}_{(0,0)}$ and $\frac{t}{u}=\frac{\beta+1}{\alpha+1} \notin \prescript{}{f'}{\mathcal{O}}_{(0,0)}$. We know that $\prescript{}{f'}{\mathcal{O}}_{(0,0)} = \prescript{}{f}{\mathcal{O}}_{(1,1)}$. That means that $a \coloneqq \frac{\alpha+1}{\beta+1}$ is the element we are looking for.

\section{}
Since we are in $\mathbb{F}_2 \implies |V_{f_2}(\mathbb{F}_2)| \leq 2^2$. So we can check all 4 possibilities for roots by substititing into $f_2(x,y)$ and we get that truly $ V_{f_2}(\mathbb{F}_2) = \{(1,0),(1,1)\}$
\begin{enumerate}[label=(\alph*)]
\item We calculate $\frac{\partial f_2}{\partial x} (x,y) = 3x^2=x^2, \frac{\partial f_2}{\partial y} (x,y) = 2y=0$ and we see that $\frac{\partial f_2}{\partial x} (1,0)=1, \frac{\partial f_2}{\partial x} (1,1) = 1$ so that means by definition that $f_2$ is smooth at $ V_{f_2}(\mathbb{F}_2)$.

Using theorem 5.8 (1) we get that there exists exactly one $P \in \mathbb{P}_{L_2/\mathbb{F}_2}$ s.t. $v_P(\alpha-1)>0,v_P(\beta)>0$. Now we can use propoisiton 5.13 (2) that tells us $P=P_{(1,0)}$. This gives us $P_{(1,0)}=P \in \mathbb{P}_{L_2/\mathbb{F}_2}$. Using the same reasoning $P_{(1,1)} \in \mathbb{P}_{L_2/\mathbb{F}_2}$.

\item By definition of $\mathcal{O}_{P_{(1,0)}}$ being a DVR, we know that there $\exists p \in P_{(1,0)}: (p)=P_{(1,0)}$. Let $P \coloneqq P_{(1,0)}$. By definition of $v_P$ we know that $p$ is a generator of $P \iff v_P(a) = 1$ because $\forall a \in P = (p): a = pb, b \in \mathcal{O}_{P}$.

By following the proof of theorem 5.8 and using the knowledge that found $P$ from 5.8 is $P_{(1,0)}$ we'll find the element $u: v_P(u)=1$. As in previous homeworks we calculate $(a_1,a_2)=(1,1)$ using partial derivatives. We choose $(b_1,b_2)=(1,0)$, that gives us $A = \begin{pmatrix}
1 & 0\\
1 & 1
\end{pmatrix}$ and $\gamma = (1,0)$. We now have $\sigma$ and we can calculate $(u,t)=\sigma(\alpha, \beta) \implies (u,t)=(\alpha+1, \alpha+\beta+1)$. We now have the generator we were looking for which is $\alpha+1$.
\end{enumerate}

\section{}
\begin{enumerate}[label=(\alph*)]
\item $\alpha^{-5} \in P \iff v_P(\alpha^{-5})\geq 1 \iff -5v_P(\alpha)\geq 1 \iff v_P(\alpha) < 0 \stackrel{\text{5.23}}{\implies} \\\exists! P \in \mathbb{P}_{L_1/\mathbb{F}_2} : v_P(\alpha)<0$. So the size of the set is 1. Assumptions of 5.23 are clearly satisfied and equivalences hold thanks to definitions and valuation properties.

\item First we will show that there does not exist a place $P \in \mathbb{P}_{L_1/\mathbb{F}_2}: \alpha \in P \land deg P = 1$. For contradiction assume there is such $P$. Using corollary 5.17 (we have shown in 2) that we can use this) we get that either $P = P_{(1,0)}$ or $P = P_{(1,1)}$ or $\alpha^{-1} \in P$. 

If $\alpha \in P \implies v_P(\alpha)\geq 1$ and if $\alpha^{-1} \in P \implies v_P(\alpha^{-1}) \geq 1 \iff v_P(\alpha) \leq -1$ which is clearly a contradiction. 

So either $P = P_{(1,0)}$ or $P = P_{(1,1)}$. In exercise 2) we have shown that $P_{(1,0)}$ is generated by $\alpha+1$ which means $P_{(1,0)}=\{(\alpha+1)r(\alpha,\beta)|r\in R_{(1,0)}\}$. So if $\alpha \in P_{(1,0)} \implies \alpha + 1 | \alpha$ which is also a contradiction. 

Using the same procedure (using the same $A$ and $\gamma =(1,1)$) as in exercise 2) we find out that $P_{(1,1)}$ is also generated by $\alpha+1$ ($P_{(1,1)}=\{(\alpha+1)r(\alpha,\beta)|r\in R_{(1,1)}\}$) and we find the same contradiction.

Therefore $\alpha$ is not in a place of degree one. Using 4.6 we get $|L:\mathbb{F}_2(\alpha)|=deg_y(f_2)=2$ and now we can use proposition 5.21 which says that if there are distinct $P_1,\dots,P_n$ where $n\in \mathbb{N}$ s.t. $v_{P_i}(\alpha)\geq 1 \iff \alpha \in P_i, \forall i=1,\dots,n$ this must hold $2\geq \sum_{i=1}^n v_{P_i}(\alpha) deg P_i$ since $v_{P_i}\geq 1$ and we have shown that $degP_i \geq 1$ means there is at most one such $P$. Using observation after 5.17 and $\alpha \in L \setminus \tilde{K}$ we know such place exists. Therefore the size of the set is 1.

\item In 2) and previous (b) we have shown that $\alpha+1 \in P_{(1,0)}, \alpha+1 \in P_{(1,1)}$. These places are distinct by definition. Also since $|L:\mathbb{F}_2(\alpha)| = 2$ and $\mathbb{F}_2(\alpha) = \mathbb{F}_2(\alpha+1)$ we get $|L:\mathbb{F}_2(\alpha+1)| = 2$. Now we use 5.21 again and we get $2\geq v_{P_{(1,0)}}(\alpha+1)degP_{(1,0)} + v_{P_{(1,1)}}(\alpha+1)degP_{(1,1)} + \dots$. We know that $v_{P_{(1,0)}}(\alpha+1) = v_{P_{(1,1)}}(\alpha+1) = 1$ and $deg P_i \geq 1$ since they are non trivial. Therefore there cannot be more places containing $\alpha+1 \implies$ the size of the set is 2.

\item Using c) we can see that $degP_{(1,0)} = degP_{(1,1)} = 1$. Since $|V_{f_2}(\mathbb{F}_2)|=2$ there can't be another place $P$ of deg 1 s.t. $P = P_\gamma$ for some $\gamma \in V_{f_2}(\mathbb{F}_2)$. The only other place of degree 1 left (according to 5.17) is $P$ s.t. $\alpha^{-1} \in P \iff -v_P(\alpha) \geq 1 \iff v_P(\alpha) \leq -1$. Corollary 5.23 tells us there is exactly one such place. Therefore the size of the set is 3.
\end{enumerate}
\end{document}

