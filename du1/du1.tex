\documentclass[12pt, a4paper]{article}
\usepackage[margin=1in]{geometry}
\usepackage[utf8x]{inputenc}
\usepackage{indentfirst} %indentace prvního odstavce
\usepackage{mathtools}
\usepackage{amsfonts}
\usepackage{amsmath}
\usepackage{amssymb}
\usepackage{graphicx}
\usepackage{enumitem}
\usepackage{subfig}
\usepackage{float}
\usepackage[czech]{babel}
\usepackage{mathdots}
\usepackage{slashbox}

\begin{document}
\begin{center}
\large NMAG436 - HW1

\normalsize Jan Oupický
\end{center}
\vspace{1\baselineskip}

\section{}
Let $w \coloneqq y^2 - f(x)$, where $f(x) \coloneqq x^3-4x^2-x+4$. By definition, we see that $w$ is a short WEP.
\begin{enumerate}[label=(\alph*)]
\item We will use the proposition 3.12 and the definition of separable polynomials. We can see that 1 and -1 are roots of $f(x)$. We can get the 3rd root by polynomial division. We now know $f(x)$ has 3 roots: $\{-1, 1, 4\}$ therefore $f(x)$ has no multiple roots because it has degree 3 (we could check this without polynomial division by checking if -1 and 1 are roots of $f'(x)$ therefore multiple roots). This means $f(x)$ is separable and by proposition 3.12 3) $w$ is smooth ($\mathbb{R}$ does not have characteristic 2).

\item From a) $f(x) = (x-1)(x+1)(x-4) \stackrel{\mathbb{F}_5}{=} (x+4)(x+1)(x+1)$. We see $4$ is a multiple root of $f(x) \implies f(x)$ is not separable. $\mathbb{F}_5$ does not have characteristic 2 so we again use proposition 3.12 3) and we get that $w$ is not smooth $\iff$ $w$ is singular.  
\end{enumerate}

\section{}
$\mathbb{R}, \mathbb{F}_3$ don't have characteristic 2.
\begin{enumerate}[label=(\alph*)]
\item Let $w \coloneqq y^2+y(2−2x)−(x^3+x^2+3x−1)$, $w$ is a WEP but not a short WEP. To apply proposition 3.12, we need the short WEP of $w$. Lemma 3.1 gives the affine bijection needed to get the short WEP form of $w$. By appling lemma 3.1 ($a = -2, c = 2$) we have: 
\[
A \coloneqq \begin{pmatrix}
1 & 0 \\
1 & 1
\end{pmatrix}, b \coloneqq \begin{pmatrix}
0 \\
-1
\end{pmatrix} \implies \sigma \coloneqq \tau_{\theta_A(b)} \circ \theta_A \implies \sigma^*(f(x,y)) = f(x, x+y-1)
\]
$\sigma^*(w) = y^2-(x^3+2x^2+x)$ is now a short WEP. Let $f(x)\coloneqq (x^3+2x^2+x)$, we can see $-1$ is a root of $f(x)$ and also a root of $f'(x)$ therefore $\sigma^*(w)$ is singular by 3.12 3) and we have only one singularity by 3.12 1) and the point is $s \coloneqq (-1,0)^T$.
Using lemma 3.10 2) we get a singular point of $w$. The point is $\sigma(s) = (-1,-2)^T$. Combination of proposition 3.12 1) and lemma 3.10 2) says there is only one singular point of $w$.

\item We will use the same steps and reasoning as in (a). Let $w \coloneqq y^2 + y(2x+1)-(x^3+2x^2+2x)$. Applying lemma 3.1 $(a=2, c=1)$:
\[
A \coloneqq \begin{pmatrix}
1 & 0 \\
2 & 1
\end{pmatrix}, b \coloneqq \begin{pmatrix}
0 \\
1
\end{pmatrix} \implies \sigma \coloneqq \tau_{\theta_A(b)} \circ \theta_A \implies \sigma^*(f(x,y)) = f(x, 2x+y+1)
\]
$\sigma^*(w) = y^2-(x^3+1)$ is a short WEP. Let $f(x) \coloneqq x^3+1$. We can guess $2$ is a root of $f(x)$ and a root of $f'(x)$. Therefore $s\coloneqq (2,0)^T$ is the only singular point of $\sigma^*(w) \implies \sigma(s) = (2,2)^T$ is the only singular point of $w$. 
\end{enumerate}

\section{}
Let $w,f$ be the same polynomials as in the 1st exercise (a). Then $W = (w)$. The roots of $f(x)$ are $\{-1,1,4\}$ therefore points $\alpha_1 \coloneqq (-1,0)^T, \alpha_2 \coloneqq (1,0)^T, \alpha_3 \coloneqq (4,0)^T \in V_w$. By lemma 4.1 $I_{\alpha_1} = (x+1,y)$. For example $y \in I_{\alpha_1}$, $y$ is irreducible in $\mathbb{R}[x,y]$ obviously $y \notin W$.

We can apply the same reasoning for the other 2 points: 

$I_{\alpha_2} = (x-1,y)$, for example $y$ is irreducible and obviously $\notin W$. 

$I_{\alpha_3} = (x-4,y)$, for example $y$ is irreducible and obviously $\notin W$.
\end{document}

